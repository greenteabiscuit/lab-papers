\documentclass[a4paper,11pt]{jsarticle}


% 数式
\usepackage{amsmath,amsfonts}
\usepackage{bm}
% 画像
\usepackage[dvipdfmx]{graphicx}


\begin{document}

\title{}
\author{}
\date{\today}
\maketitle

\section{tangent delta}

\section{evanescent}

\begin{equation}
  z_g = \frac{c}{\omega\sqrt[]{\epsilon_1\mu_1\sin^{2}\theta - \epsilon_2\mu_2}}
\end{equation}

$z_g$ は第二媒質への電界の侵入深さを表し、電磁波は境界面へわずかにしみでていることがわかる。
全反射時にも電磁波は境界面の反対側にもわずかに浸みだす。
この成分をevanescent波と呼ぶ。
電磁界の浸みこみと合わせて、反射点もずれる。
このずれのことをグースヘンヒェンシフトと呼ぶ。
誘電体導波路、光ファイバーを利用する際に重要となる。


\end{document}