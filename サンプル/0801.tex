\documentclass[a4paper,12pt]{report}
% use pdflatex
\usepackage[utf8]{inputenc}
\usepackage{CJKutf8}

\begin{document}
\thispagestyle{plain}
\begin{center}
    \Large
    \begin{CJK}{UTF8}{ipxm}論文要旨\end{CJK}

    \vspace{2.0cm}

    \Large
    \textbf{Research on Dielectric Waveguides for Enhancing Spatial Resolution in Beyond 5G High Frequency Communications}
        
    \vspace{0.4cm}
    \large
        
    \vspace{0.4cm}
    \begin{CJK}{UTF8}{ipxm}三谷怜司\end{CJK}
       
    \vspace{0.9cm}
    \textbf{Abstract}
\end{center}

With the rapid development of communication technology, the demand
for antennas and cables, essential components for signal processing,
is growing by leaps and
bounds.
In addition,
demand for mmWave antennas and cables is increasing as
components for specific bands are necessary for individual
merchants to build their own Local 5G base stations.

However, components such as antennas and cables used
currently need certain
processing costs and processing time,
increasing the complexity and spectral characteristics,
and creating limitations in the design and implementation.
Moreover, it takes
more time and expense for specific antennas and cables
due to their particular
characteristics.
Furthermore, infrastructure created by traditional components
is often irreversible, and cannot be changed once
the components are set in position.
This is acceptable for a small number of network areas, but if the number
of local networks is large and the architecture is complex,
this leads to interference between wireless signals.
For this reason, we propose to design and produce
mmWave antennas using dielectric waveguides to reduce the cost,
processing time
and create a distributive local area network.

This paper mainly designs three types of dielectric waveguide antennas
and details the design process and the simulation process.
First, we introduce the related background
and how Maxwell's electromagnetic equations tie into the characteristics
of dielectric waveguides.
Then we describe the basic theory of the antenna design
and simulate them on a simulator.
After the structure and order are determined first,
we iterate the simulation until
the S-parameters and half-power beam width are optimized.
After
assembling the dielectric waveguide antenna,
we observe the actual
test results and demonstrate the feasibility
using the scalar network analyzer and the vector network analyzer.

The implemented antennas shows the antenna functions with less than 10dB
insertion loss and around 20 degrees of half-power beamwidth.
The proposed implementation
method can effectively reduce the machining time in 1 day
due to the simple principle and the procedural process.
Furthermore, the high-resolution beam
emitted from the tip of the antenna is narrow enough so
that multiple network areas can be built without interfering with each other.

These advantages will also enable Local 5G operators
to fulfill the customers' requirements within a short period. In terms of
processing materials,
it can effectively reduce the cost of goods by at least
80\%.


\end{document}