\documentclass[a4paper,12pt]{report}
% use pdflatex
\usepackage[utf8]{inputenc}
\usepackage{CJKutf8}

\begin{document}
\thispagestyle{plain}
\begin{center}
    \Large
    \begin{CJK}{UTF8}{ipxm}論文要旨\end{CJK}

    \vspace{2.0cm}

    \Large
    \textbf{Research on Dielectric Waveguides for Enhancing Spatial Resolution in Beyond 5G High Frequency Communications}

    \vspace{0.4cm}
    \begin{CJK}{UTF8}{ipxm}Beyond 5G 高周波通信の空間分解能を向上するための誘電体導波路の研究\end{CJK}
        
    \vspace{0.4cm}
    \large

    \vspace{0.4cm}
    \begin{CJK}{UTF8}{ipxm}総合分析情報学コース\end{CJK}

    \vspace{0.4cm}
    \begin{CJK}{UTF8}{ipxm}49-216421\end{CJK}

    \vspace{0.4cm}
    \begin{CJK}{UTF8}{ipxm}三谷 怜司\end{CJK}
       
    \vspace{0.9cm}
\end{center}

\begin{CJK}{UTF8}{ipxm}
5G・Beyond5Gで普及が期待されている高周波数の通信では、帯域を広く利用することで大容量の通信が可能となる一方、電波伝播の直進性や急激な減衰から利活用の困難さが指摘されている。我々は、ミリ波などの高周波数を利用する通信において、直進性や減衰性を逆に利用し、高空間分解能の通信に利活用することを考えている。一般に、ミリ波ではアレイアンテナを利用するビームフォーミングにより電波伝播の範囲を限定することが可能であるが、コストが高いなどの課題がある。本研究では、低コストでビームを絞るための誘電体導波路を用いたミリ波アンテナを設計・製作することを提案し、シミュレーション評価によりその有用性を示す。
\end{CJK}

\vspace{1.0cm}

High-frequency communications, which are expected to be widely used in 5G and Beyond5G, enable high-capacity communications by using a wide bandwidth, but the linearity of radio wave propagation and rapid attenuation have made it difficult to utilize them. We are considering the use of millimeter waves and other high-frequency communications for high-spatial-resolution communications by taking advantage of their linearity and attenuation. Generally, it is possible to limit the range of radio propagation by beamforming using array antennas in millimeter-wave communications, but the cost is high. In this study, we propose to design and fabricate a millimeter-wave antenna using a dielectric waveguide to narrow the beam at low cost, and demonstrate its usefulness through simulation evaluation.

\end{document}
